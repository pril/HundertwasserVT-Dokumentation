\documentclass[a4paper,final,12pt]{scrreprt}
\usepackage[utf8]{inputenc}
\usepackage[T1]{fontenc}
\usepackage{marginnote}
\usepackage[ngerman]{babel}
\usepackage{amsmath}
\usepackage{color}
\usepackage{fancyhdr}
\usepackage{bibgerm}
\usepackage{cite}
\usepackage[justification=RaggedRight, singlelinecheck=false]{caption} 
\usepackage{graphicx}
\usepackage{listings}

\usepackage[
    bookmarks=true,         % show bookmarks bar?
    unicode=false,          % non-Latin characters in Acrobat’s bookmarks
    pdftoolbar=true,        % show Acrobat’s toolbar?
    pdfmenubar=true,        % show Acrobat’s menu?
    pdffitwindow=false,     % window fit to page when opened
    pdfstartview={FitH},    % fits the width of the page to the window
    pdftitle={Projektdokumentation Hundertwasser},    % title
    pdfauthor={Daniel Rhein},     % author
    pdfsubject={Hundertwasser},   % subject of the document
    pdfcreator={Daniel Rhein},   % creator of the document
    pdfproducer={Daniel Rhein}, % producer of the document
    pdfkeywords={keyword1} {key2} {key3}, % list of keywords
    pdfnewwindow=true,      % links in new window
    colorlinks=false,       % false: boxed links; true: colored links
    linkcolor=black,          % color of internal links (change box color with linkbordercolor)
    citecolor=black,        % color of links to bibliography
    filecolor=black,      % color of file links
    urlcolor=black           % color of external links
]{hyperref}

\title{Projektdokumentation\\Verteilte Systeme}

\author{Daniel Rhein}
\date{20.03.2012}
\begin{document}
 \pagestyle{empty}
\maketitle
%Später titelblatt erstellen

\pagestyle{fancy}
\cfoot{}
\rfoot{Seite \thepage}
\tableofcontents
\setcounter{chapter}{1}
\newpage
\section{Dokument History}
	\begin{table}[h]
	\begin{tabular}{|l|l|l|r|}\hline
	Datum & Beschreibung & Bemerkung & Autor \\\hline
	20.03.2013 & Draft & \LaTeX{} Dokument erstellt & Daniel Rhein \\\hline
	20.03.2013 & Einarbeitung & Dokumentinhalt erstellt	 & Daniel Rhein \\\hline
	\end{tabular}
	\caption[Dokumenthistorie]{Dokumenthistorie}  
	\end{table}

\section{Zweck des Dokuments}
\marginnote{20.03.2013 09:16 Uhr}
In diesem Dokument soll das aktuelle Projekt und dessen Umsetzung sowie wichtige Entscheidungen Dokumentiert werden.
\section{Aufgabe}
Inhalte zusammenfassen und Dokument erstellen.
\section{Socket}
\section{RMI}
\section{EJB}
\section{JBOSS Application Server}
\section{Clustering}

\listoffigures
\listoftables
\bibliographystyle{geralpha}
%\phantomsection 
%\addcontentsline{toc}{chapter}{Literatur} 
\bibliography{literaturverzeichnis} 
\section{Appendix }
\end{document}